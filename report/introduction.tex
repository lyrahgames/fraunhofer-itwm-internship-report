\documentclass[crop=false]{standalone}
\usepackage{standard}
\begin{document}
  \section{Einleitung} % (fold)
  \label{sec:einleitung}
    In der Zeit zwischen dem 26.~November 2013 und dem 30.~Juni 2017 arbeitete ich am Fraunhofer ITWM im Competence Center High Performance Computing (CC HPC) als Hilfswissenschaftler.
    Im Verlauf dieser Beschäftigung wurde mir die Möglichkeit zu Teil, mich mit diversen Bereichen der Informatik und der Mathematik, wie zum Beispiel der Programmoptimierung, dem Parallel Computing, der Computergrafik und den Monte-Carlo-Methoden, sowie auch einigen Standardwerkzeugen der Softwareentwicklung, wie zum Beispiel \textit{Git}, \textit{Make}, \textit{CMake}, \textit{OpenGL}, \textit{Qt}, \textit{CUDA} und \textit{Blender}, genauer auseinanderzusetzen.
    Vor allem für meine eigentliche Tätigkeit, der Entwicklung von Softwarelösungen für vorgegebene Problemstellungen, zeigten sich die erlangten Fachkenntnisse und Erfahrungen in diesen Gebieten von hohem Nutzen und ermöglichten mir eine professionelle Arbeitsweise innerhalb der verschiedenen Projektgruppen.
    Die komplexesten Aufgabenstellungen bestanden dabei aus der Implementierung eines echtzeitfähigen Raytracers auf der GPU und CPU, der Konstruktion eines statistisch-basierten graphischen Analysewerkzeugs für seismische Daten und der Entwicklung einer Klasse, die das Einlesen des Wavefront OBJ Dateiformates ermöglicht.
    Für die Ausführung der genannten Aufgaben verwendete ich im Wesentlichen die Programmiersprachen \textit{C} und \textit{C++}.
  % section einleitung (end)
\end{document}