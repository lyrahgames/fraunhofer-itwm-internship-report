\documentclass[crop=false]{standalone}
\usepackage{standard}
\begin{document}
  \section{Selbsteinschätzung und Bezug zum Studium} % (fold)
  \label{sec:selbsteinschaetzung}
    Im Rahmen des Projektes konnte ich tiefgreifende Kenntnisse in der Anwendung von den in Abschnitt \ref{sec:tools} beschriebenen Hilfsmitteln erwerben.
    Diese habe ich mir selbst angeeignet und zudem in der Praxis getestet.
    Dadurch konnte ich das erlangte Wissen nicht nur auf vereinzelte Projekte sondern auch auf die vielfältigen Kontexte des Studiums anwenden.
    Als Beispiel seien hier diverse Computational Science Praktika genannt, bei denen ich durch die Anwendung von \textit{Git}, \textit{CMake} und \textit{ClangFormat} eine wesentlich effizientere Arbeitsweise, sowie auch besser strukturierten beziehungsweise lesbaren Quelltext, entwickelte.
    Des Weiteren lernte ich, mir eine Entwicklungsumgebung auf verschiedenen Computersystem einzurichten und zu warten.
    Beim Einrichten dieser achtete ich darauf, dass die Verwendung des aktuellen Standards der verschiedenen Tools möglich war.
    % Folglich bin ich der absolute Obermacker, was Standards angeht, Alter!
    Folglich besitze ich einen guten Überblick über die Standards von \textit{C++}, \textit{Git} und \textit{CMake}.
    Diese Kompetenzen konnte ich im Studium vor allem dafür verwenden, anderen Studenten zu helfen und mich in Gruppen besser zu etablieren.
    Gerade durch die Zusammenarbeit innerhalb einer Gruppe erkannte ich, wie wichtig ein sehr gut lesbarer, strukturierter und konsistenter Programmierstil ist.
    Während des Projektes konnte ich mir diesen aneignen und in darauffolgenden Projekten verbessern.
    Somit diente der entstehende Code zu jeder Zeit als solides Fundament für die Entwicklung des Projekts.
    Erst dadurch war es möglich die Problemstellungen kreativ und effizient zu lösen und flexibel auf die Anforderungen anderer Teammitglieder einzugehen, ohne sich in der Unordnung des Quelltextes zu verlieren.

    Innerhalb des Projektes gab es diverse Möglichkeiten die aus dem Studium bekannte Mathematik anzuwenden.
    Es waren bereits grundlegende Kenntnisse der linearen Algebra nötig, um den naiven Algorithmus, der für alle weiteren Optimierungen als Ausgangspunkt diente, zu konstruieren.
    Für die grafische Ausgabe und die Schnittpunktberechnungen waren die Matrix- und Vektorrechnung unerlässlich.
    Zudem war es im Computer nicht möglich, echte reelle Zahlen darzustellen.
    Als Ersatz wurden die aus der Numerik bekannten Gleitkommazahlen verwendet, deren endliche Genauigkeit zu Fehlern und Artefakten führen kann.
    Erst durch den Gebrauch numerischer Methoden konnten diese Fehler nicht nur verstanden, sondern auch behoben werden.
    Auch die Konstruktion von Beschleunigungsstrukturen, wie der BVH, konnte durch mathematische Optimierung erklärt und teilweise verbessert werden.
    Dies lässt sich dadurch erklären, dass Beschleunigungsstrukturen für einen Raytracer immer auf ein Minimierungsproblem zurückzuführen sind.
    Die Berechnung der Leuchtdichte-Verteilung im Raum basiert auf physikalischen Theoremen der geometrischen Optik, wie zum Beispiel der Rendergleichung%
    \footnote{https://de.wikipedia.org/wiki/Rendergleichung}.
    Die verschiedenen Materialien der Objekte lassen sich mithilfe der Elektrodynamik beschreiben und können durch ihre speziellen Eigenschaften die Auswertung der Leuchtdichte unter Umständen vereinfachen.
    Die eigentliche Berechnung der Leuchtdichte kann sogar auf zwei Gebiete der Mathematik, der höheren Analysis und der Stochastik, zurückgeführt werden.
    Die Berechnung der Leuchtdichte ist äquivalent zu dem Lösen einer Integralgleichung.
    Sätze der höheren Analysis erlauben dem Entwickler das Problem in eine einfachere Form zu transformieren.
    Durch stochastische Methoden, wie der Monte-Carlo Methode, wird man dann in die Lage versetzt das Problem erwartungstreu zu lösen.
  % section selbsteinschätzung (end)
\end{document}